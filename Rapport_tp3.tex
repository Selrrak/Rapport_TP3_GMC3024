% Basic setup
\documentclass[12pt,a4paper]{article} % Defines the document type as an article with 12pt font size on A4 paper.

% Importing packages
\usepackage[english]{babel} % Sets the document language to English, adjusting hyphenation and language-specific typographic rules.
\usepackage[lmargin=2.5cm,rmargin=2.5cm,tmargin=2.5cm,bmargin=2.5cm]{geometry} % Sets custom page margins: left/right 2.5cm, top 2.5cm, bottom 2.5cm.

% Loading packages
\usepackage{amsmath}
\usepackage{hyperref}    % Enables hyperlinks for references, URLs, and citations.
\usepackage{xcolor}      % Provides tools for defining and using colors.
\usepackage{graphicx}    % Allows inclusion of images and graphics.
\usepackage{caption}     % Customizes captions for figures and tables.
\usepackage{subcaption}  % Supports subfigures and subcaptions within figures.
\usepackage{minted}      % Enables syntax highlighting for code listings.
\usepackage[T1]{fontenc} % Ensures proper font encoding, important for correct character rendering. Don't touch this.
\usepackage{lmodern}     % Sets the font to something more modern and easy to read.
\usepackage{setspace}    % Provides control over line spacing.
\usepackage{csquotes}    % Improves handling of quotations.
\usepackage{setspace}    % Used to set line spacing
\usepackage{longtable,booktabs,array} % Packages for advanced table formatting.
\usepackage[
  backend=biber,
  style=apa,  
]{biblatex}  % Manages citations and bibliography with APA style.

\setstretch{1.5} % Sets line spacing

\definecolor{LightGray}{gray}{0.9}  % Defines a custom color 'LightGray' with 90% gray, used for the code block background.
\hypersetup{                        % Configures hyperlink colors and behavior.
  colorlinks=true,                  % Enable colored links instead of boxes.
  linkcolor={blue},                 % Sets link color to blue.
  filecolor={maroon},               % Sets file link color to maroon.
  citecolor={blue},                 % Sets citation link color to blue.
  urlcolor={blue}}                  % Sets URL link color to blue.

\addbibresource{assets/bib-template.bib} % Adds the bibliography file.

\title{Place Holder page titre}        % Sets the document title.
\makeatletter
\providecommand{\subtitle}[1]{%     % Custom command to add a subtitle.
  \apptocmd{\@title}{\par {\large #1 \par}}{}{}
}

\makeatother
\subtitle{Va être remplacée par celle sur Teams}  % Sets the document subtitle.
\author{Charles Bouthillier Paul Charvet William Hamilton Samuel Roy}% Sets the author's name.
\date{2025-11-06}                   % Sets the document date.

% \renewcommand{\familydefault}{\sfdefault} % Changes the default font to one without serifs

% From this point, the preamble ends and the actual content of the document starts.
\begin{document}
\pagenumbering{gobble} % Stops counting the pages from this point until changed again.
\maketitle
%\begin{abstract}
%This is the LaTeX template (\textbf{Version 1.1}) from the DH Lab of the University of Basel. It is suitable for Seminar Papers and Master's / PhD theses, but can be adapted to fit a variety of use cases. It was created by Stefan Freitag and is derived from the Quarto template, which is also available from the DH Lab.

%The abstract of your paper / thesis goes right here. It will appear on the cover sheet (very first page) of the PDF, together with the title, subtitle, author name and date.

%If you do not want your document to have an abstract, you can also simple delete the 'abstract' part (or comment it out) with all its content and the final document will be created without it.
%\end{abstract}
\begin{center}
    \vfill
    \begin{figure}[H] 
          \includegraphics[width=.8\linewidth]{./assets/Université_Laval_logo_et_texte.svg.png}
    \end{figure}
        \setcounter{figure}{0}
        
    Université Laval\\
    Facutlé de science génie\\
    Québec
\end{center}
\newpage
\renewcommand*\contentsname{Table des matières} % This controls the title of your table of contents.
{
\hypersetup{linkcolor=}
\setcounter{tocdepth}{5} % Sets the maximum sublevel to be displayed within the table of contents.

\tableofcontents
}
\newpage
\pagenumbering{arabic}\setstretch{1.5} % Overwrites the previous command, pages are counted as normal from this point.


\section{Vue CAD 3D explosée}
    \begin{figure}[H] 
	    \includegraphics[width=1.1\linewidth]{./assets/Explose.pdf}
    \end{figure}

\section{captures d'écran des deux enveloppes d'impression}

\subsection{Volume Préférentiel X-Y}
    \begin{figure}[H] 
	    \includegraphics[width=1.1\linewidth]{./assets/XY_pref.pdf}
    \end{figure}

    \subsection{Volume Préférentiel Z}
    \begin{figure}[H] 
	    \includegraphics[width=1.1\linewidth]{./assets/Z_pref.pdf}
    \end{figure}

\section{Dessin de fabrication du corps de pompe}
    \begin{figure}[H] 
	    \includegraphics[width=1.1\linewidth]{./assets/fabrication.pdf}
    \end{figure}
\section{rapport PolyWorks sur tolérance géométrqiue de l'axe du levier}
    \begin{figure}[H] 
	    \includegraphics[width=1.0\linewidth]{./assets/Polyworks.pdf}
    \end{figure}
\section{Calculs}
\subsection{Joint d'étanchéité piston-cylindre de pompe}
\subsubsection{Schéma}
    \begin{figure}[H] 
	    \includegraphics[width=1.1\linewidth]{./assets/Oring_scheme.png}
    \end{figure}
\subsubsection{Données techniques}
\begin{center}
	\begin{tabular}{|c c c|}
		\hline
		Paramètre & symbole & valeur de base\\
		\hline
		Coefficient de Poisson du caouchouc & $\mu$ & 0,5\\
		Pourcentage d'étirement du joint torique & Stretch\% & 5\%\\
		Pourcentage de réduction de la section du joint torique & AR\% & N/A\\
		Diamètre de la section du joint torique & W & 2,62 mm\\
		Diamètre de la section du joint torique étiré & WR & N/A\\
		Profondeur de la rainure, incluant jeu diamétrale & F & N/A\\
		Compression du joint torique & SQ\% & 21\%$\in$[12;24]\%\\
		Diamètre interne du reservoir & BORE & 38 mm\\
		Diamètre du fond de rainure sur le piston & PG & N/A\\
		Diamètre intérieur du joint torique & ID & N/A\\
		Volume du joint torique & OVol & N/A\\
		Pourcentage de remplissage de la rainure & Fill\% & 65\%\\
		Volume de la rainure & GVol & N/A\\
		Largeur de la rainure & G & N/A\\
		\hline
	\end{tabular}
\end{center}
\subsubsection{Équations}
\begin{equation}
	AR\% = Stretch\% * \mu 
\end{equation}
\[ AR\% = 5\% * 0,5 = 2,5\% \]

\begin{equation}
	WR = W - (\frac{AR\%}{100})*W 
\end{equation}
\[WR = 2,62 - (\frac{2,5\%}{100})*2,62 = 2,56 mm\] 


\begin{equation}
	SQ\% = (\frac{WR - F}{W})*100\% 
\end{equation}

\[ F = (WR - \frac{SQ\%}{100\%})*W \]

\[ F = (2,56 - \frac{21\%}{100\%})*2,62 = 2,00 mm  \]

\begin{equation}
	F = (\frac{BORE-PG}{2}) 
\end{equation}
\[PG = BORE - 2F\]
\[PG = 38 - 2*2 = 34 mm\]

\begin{equation}
	Stretch\% = (\frac{PG -ID}{ID})*100\%
\end{equation}
\[ ID = \frac{PG}{(\frac{Stretch\%}{100})+1} \] 
\[ ID = \frac{34}{(\frac{5\%}{100})+1} = 32,38 mm \]

\begin{equation}
	OVol = \frac{(\pi)^2}{4}*(ID+W)*W^2
\end{equation}

\[ OVol = \frac{(\pi)^2}{4}*(32,38+2,62)*2,62^2 = 592,82 mm^3 \] 

\begin{equation}
	Fill\% = \frac{OVol}{GVol}*100\%
\end{equation}
\[ GVol = \frac{100\%*OVol}{Fill\%} \]
\[ GVol = \frac{100\%*592,82}{65\%} = 912,03 mm^3 \] 

\begin{equation}
	GVol = \frac{\pi}{4}*(BORE^2 - PG^2)*G
\end{equation}
\[ G = \frac{4*GVol}{\pi*(BORE^2 - PG^2)} \]
\[ G = \frac{4*912,3}{\pi*(38^2 - 34^2)} = 4,32 mm \]

\subsection{Pièce encliquetée}
\subsubsection{Schémas}
\begin{figure}[H]
    \centering
    
    \begin{subfigure}[t]{0.48\linewidth}
        \centering
        \includegraphics[height= 10 cm, width=\linewidth]{./assets/Pump_scheme.png}
        \caption{Références des dimensions}
    \end{subfigure}
    \hfill
    \begin{subfigure}[t]{0.48\linewidth}
        \centering
	\includegraphics[height = 10 cm, width=\linewidth]{./assets/Pump_DCL.png}
        \caption{DCL}
    \end{subfigure}
    \caption{Pompe entière}
\end{figure}

\begin{figure}[H]
    \centering
    \includegraphics[width=\linewidth]{./assets/Piston_valve.png}
    \caption{Valve du piston}
\end{figure}

\begin{figure}[H]
    \centering
    \includegraphics[width=\linewidth]{./assets/Lever_DCL.png}
    \caption{DCL et dimensions du Bras de levier}
\end{figure}

\begin{figure}[H]
    \centering
    
    \begin{subfigure}[t]{0.48\linewidth}
        \centering
        \includegraphics[height=9 cm, width=\linewidth]{./assets/Click_scheme.png}
        \caption{Références des dimensions}
    \end{subfigure}
    \hfill
    \begin{subfigure}[t]{0.48\linewidth}
        \centering
        \includegraphics[height= 9 cm, width=\linewidth]{./assets/Click_DCL.png}
        \caption{DCL}
    \end{subfigure}
    \caption{Pièce encliquetée}
\end{figure}

\subsubsection{Questions techniques et description des calculs}
	\begin{enumerate}
		\item La pièce encliquetée cède-t-elle lorsque l'utilisateur actionne la pompe? \par
			La charge nécessaire au flambage de la pièce ainsi que la contrainte de rupture en compression du matériau seront comparer à la force/contrainte subit par la pièce encliquetée. Celle-ci est liée à celle pour faire bouger le piston par l'entremise du bras de levier. Un diagramme de corps libre permet de trouver ce lien. Quant à la force pour faire bouger le piston, on l'estime maximale lorsque le piston est dans sa phase de descente, juste avant que les valves qui le traverse ne s'ouvrent. Si on suppose que le piston est parfaitement étanche, on peux alors supposer que la pression au dessus du piston est nulle, alors que celle sous celui-ci est égale à la pression de craquage de ses valves plus la friction entre les joints toriques et le bâti de la pompe. Cette dernière est évalué à l'aide du pourcentage de compression des joints toriques, alors que la pression de craquage est une fonction de la précharge sur les ressorts des valves à bille. 
		\item La pièce cède-t-elle sous la déformation causée par l'encliquetage? \par
			Calculer la déformation de la pièce encliquetée et la comparer à la déformation maximale possible du matériau en fonction de son orientation d'impression.
	\end{enumerate}
\subsubsection{Données techniques}
\begin{center}
	\begin{tabular}{|c c c|}
		\hline
		Paramètre & symbole & valeur de base\\
		\hline
		Déformation à la rupture & $\varepsilon_{ruptureXY}$ & 16\%\\
		Déformation maximale & $\varepsilon_{max}$ & N/A\\
		Longueur des faisceaux & L & 11 mm\\
		Diamètre au crochet & D1 & 8 mm\\
		Diamètre de la pièce encliquetée & D2 & 6 mm\\
		Espacement entre les deux faisceaux & d & 2,2 mm\\
		Largeur d'un faisceau & $h_{0}$ & N/A\\
		Déflection maximale des faisceaux & Y & N/A\\
		Déformation lors de l'encliquetage & $\varepsilon$ & N/A\\
		Facteur de sécurité & n & N/A\\
		\hline
	\end{tabular}
\end{center}

\newpage{}

\subsubsection{Équations}
\begin{equation}
	\varepsilon_{\text{max}} = 0.2\,\varepsilon_{\text{ruptureXY}}}
\end{equation}
\[ \varepsilon_{max} = 20\% * 16\% = 3,2\% \]

\begin{equation}
	h_{0} = \frac{D2-d}{2}
\end{equation}
\[ h_{0} = \frac{6-2,2}{2} = 1,9 mm \]

\begin{equation}
	Y = \frac{d}{2}
\end{equation}
\[ Y = \frac{2,2}{2} = 1,1 mm \]

\begin{equation}
	\varepsilon = \frac{3*Y*h_{0}}{2*L^2}
\end{equation}
\[ \varepsilon = \frac{3*1,1*1,9}{2*11^2} = 2,59\% \]

\begin{equation}
	n = \frac{\varepsilon}{\varepsilon_{max}}
\end{equation}
\[ n = \frac{2,59}{3,2} = 1,235 > 1 \]

\newpage{}

\printbibliography[ % Prints the bibliography
heading=bibintoc,
title={References} % title of the 'references' section, change this if necessary
]

\end{document}
