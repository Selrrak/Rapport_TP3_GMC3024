% Basic setup
\documentclass[12pt,a4paper]{article} % Defines the document type as an article with 12pt font size on A4 paper.

% Importing packages
\usepackage[english]{babel} % Sets the document language to English, adjusting hyphenation and language-specific typographic rules.
\usepackage[lmargin=2.5cm,rmargin=2.5cm,tmargin=2.5cm,bmargin=2.5cm]{geometry} % Sets custom page margins: left/right 2.5cm, top 2.5cm, bottom 2.5cm.

% Loading packages
\usepackage{hyperref}    % Enables hyperlinks for references, URLs, and citations.
\usepackage{xcolor}      % Provides tools for defining and using colors.
\usepackage{graphicx}    % Allows inclusion of images and graphics.
\usepackage{caption}     % Customizes captions for figures and tables.
\usepackage{subcaption}  % Supports subfigures and subcaptions within figures.
\usepackage{minted}      % Enables syntax highlighting for code listings.
\usepackage[T1]{fontenc} % Ensures proper font encoding, important for correct character rendering. Don't touch this.
\usepackage{lmodern}     % Sets the font to something more modern and easy to read.
\usepackage{setspace}    % Provides control over line spacing.
\usepackage{csquotes}    % Improves handling of quotations.
\usepackage{setspace}    % Used to set line spacing
\usepackage{longtable,booktabs,array} % Packages for advanced table formatting.
\usepackage[
  backend=biber,
  style=apa,  
]{biblatex}  % Manages citations and bibliography with APA style.

\setstretch{1.5} % Sets line spacing

\definecolor{LightGray}{gray}{0.9}  % Defines a custom color 'LightGray' with 90% gray, used for the code block background.
\hypersetup{                        % Configures hyperlink colors and behavior.
  colorlinks=true,                  % Enable colored links instead of boxes.
  linkcolor={blue},                 % Sets link color to blue.
  filecolor={maroon},               % Sets file link color to maroon.
  citecolor={blue},                 % Sets citation link color to blue.
  urlcolor={blue}}                  % Sets URL link color to blue.

\addbibresource{assets/bib-template.bib} % Adds the bibliography file.

\title{Place Holder page titre}        % Sets the document title.
\makeatletter
\providecommand{\subtitle}[1]{%     % Custom command to add a subtitle.
  \apptocmd{\@title}{\par {\large #1 \par}}{}{}
}

\makeatother
\subtitle{Va être remplacée par celle sur Teams}  % Sets the document subtitle.
\author{Charles Bouthillier Paul Charvet William Hamilton Samuel Roy}\\% Sets the author's name.
\date{2025-11-06}                   % Sets the document date.

% \renewcommand{\familydefault}{\sfdefault} % Changes the default font to one without serifs

% From this point, the preamble ends and the actual content of the document starts.
\begin{document}
\pagenumbering{gobble} % Stops counting the pages from this point until changed again.
\maketitle
%\begin{abstract}
%This is the LaTeX template (\textbf{Version 1.1}) from the DH Lab of the University of Basel. It is suitable for Seminar Papers and Master's / PhD theses, but can be adapted to fit a variety of use cases. It was created by Stefan Freitag and is derived from the Quarto template, which is also available from the DH Lab.

%The abstract of your paper / thesis goes right here. It will appear on the cover sheet (very first page) of the PDF, together with the title, subtitle, author name and date.

%If you do not want your document to have an abstract, you can also simple delete the 'abstract' part (or comment it out) with all its content and the final document will be created without it.
%\end{abstract}
\begin{center}
    \vfill
    \begin{figure} 
          \includegraphics[width=.8\linewidth]{./assets/Université_Laval_logo_et_texte.svg.png}
    \end{figure}
        \setcounter{figure}{0}
        
    Université Laval\\
    Facutlé de science génie\\
    Québec
\end{center}
\newpage
\renewcommand*\contentsname{Table des matières} % This controls the title of your table of contents.
{
\hypersetup{linkcolor=}
\setcounter{tocdepth}{5} % Sets the maximum sublevel to be displayed within the table of contents.

\tableofcontents
}
\newpage
\pagenumbering{arabic}\setstretch{1.5} % Overwrites the previous command, pages are counted as normal from this point.


\section{Vue CAD 3D explosée}
\begin{itemize}
	\item vue isométrique
	\item Fil de fer noir ou blanc
	\item lignes cachées supprimées
	\item Item flèche-bulle numéroté pour chaque pièce
	\item Tableau de nomenclature (#bulle, qté, description)
	\item Indiquer le diamètre intérieur à usiner
	\item Indiquer sur une note le nombre total de pièces
\end{itemize}

\section{captures d'écran des deux enveloppes d'impression}
\begin{itemize}
	\item affichages des dimensions limites
	\item arrangement des pièces 3D à l'intérieur
\end{itemize}
\subsection{Volume Préférentiel X-Y}
\subsection{Volume Préférentiel Z}


\section{This is a section}

\subsection{This is a sub-section}

\subsubsection{This is a sub-sub-section}

\paragraph{This is a paragraph}

\subparagraph{This is a sub-paragraph}

Up to 5 sublevels are possible, as you can see. By default, the sections are automatically numbered while the (sub)-paragraphs are not. This of course can be changed by you. The table of contents will be automatically generated (including said numbering and page numbers) upon compiling your document by using the using the \texttt{\string\tableofcontents} command. \textbf{You do not need to write it yourself.}

See \href{https://www.overleaf.com/learn/latex/Sections_and_chapters}{this} part of the tutorial for more information about sections and chapters.

\section{Formatting your document}

LaTeX supports a variety of formatting options:

\emph{You can write in italics.}

\textbf{Or bold.}

\textbf{\emph{Or both at the same time.}}

You can also superscript\textsuperscript{text} or subscript\textsubscript{text}.

You can also write lists (unordered or ordered):

\begin{itemize}
\item
    Example list
\item 
    This one is unordered in the form of bullet points
\end{itemize}

\begin{enumerate}
\item
    This
\item
    is
\item
    an
\item
    ordered
\item
    list
\end{enumerate}

and use tables (with various formatting options, this is just an example):

\begin{table}[h!]
\centering
\begin{tabular}{|c c c|} 
 \hline
 Col1 & Col2 & Col3 \\ [0.5ex] 
 \hline\hline
 1 & 2 & 3 \\ 
 2 & 3 & 4 \\
 3 & 4 & 5 \\
 4 & 5 & 6 \\
 5 & 6 & 7 \\ [1ex] 
 \hline
\end{tabular}
\caption{Example table}
\end{table}

This is a more complex table, but it shows what is theoretically possible. As you can see, the tables are automatically numbered as well, as long as you give them a caption.

It is possible to automatically generate a list of tables at any point of the document. To do this, use the command \texttt{\string\listoftables}:

\listoftables

You can also insert page breaks like this:

\newpage{}

These are only some examples, please see the respective sections in the tutorial for more: \href{https://www.overleaf.com/learn/latex/Bold%2C_italics_and_underlining}{text formatting}, \href{https://www.overleaf.com/learn/latex/Lists}{lists} and \href{https://www.overleaf.com/learn/latex/Tables}{tables}.

\section{Using images}\label{using-images}

With LaTeX, images are embedded within the text with a caption and a
path to the image. This requires the images to be a part of your project and uploaded to Overleaf. In this template, all images go into the \texttt{assets} folder.

This is how an image is inserted into the document:

\begin{figure}[H]

{\centering % Command to center the image.

\includegraphics[width=\textwidth]{assets/dhlab-logo-black.png} % Logo is scaled to be exactly as wide as the text.

}

\caption{Logo of the DH Lab}

\end{figure}

You can also resize images:

\begin{figure}[H]

{\centering 

\includegraphics[width=0.5\textwidth]{assets/dhlab-logo-black.png} % Scaled to 0.5 (50%) of the width of the text.

}

\caption{Logo of the DH Lab, but smaller}

\end{figure}

As you can see, images with captions are numbered automatically. Like with tables, it is
possible to generate a list of figures at any point of the document.
To do this, use the command \texttt{\string\listoffigures}:

\listoffigures

Once again, check the corresponding
\href{https://www.overleaf.com/learn/latex/Inserting_Images}{tutorial section} for
further information!

\section{Using code}

When writing a paper or thesis in the field of Digital Humanities, it is likely that you will need to also use code within your document. This is possible with LaTeX, if you are working with a lot of code however and / or want it to be executable within the document file, it is recommended to use the \href{https://github.com/SteFre95/Digital_Humanities_Lab_UniBas_Quarto_Template}{Quarto template} of the DH Lab instead. This also enables you to easily include the output of your code in a document. It is, however, more complex to set up, get running and work with.

In order to properly display code including syntax highlighting, this template employs the \texttt{minted} package, which works like this:

\begin{minted}
[
bgcolor=LightGray
]
{python}

# This is a code cell written and highlighted with the minted package

for n in range(3):
    print('This is an example for a Python code cell')
\end{minted}

There are many ways to customize how the code is displayed, refer to \href{https://www.overleaf.com/learn/latex/Code_Highlighting_with_minted}{this} section of the Overleaf tutorial for further information. There, you will also see how to include code written in a separate file.

Keep in mind that you need to manually include the output of the code in the document, should this be important to you. The LaTeX compile process does \textbf{not} run the code in any way.

\section{Citations and References}

Citations and references to literature and other things work like this:

Authors X and Y say a lot of things about many topics (\cite{abdullahlajamPerformanceEvaluationIPFS2021}).

These guys also talk about interesting stuff on certain pages (\cite[pp. 136-139]{alniamyAttributebasedAccessControl2020}).

You can also do citations that flow naturally with the text, like this: According to \textcite{abdullahlajamPerformanceEvaluationIPFS2021}, topic X is really cool.

This style of citation (footnotes) is also possible\footnote{This guy (\cite{gauthierReligionModernityGlobalisation2019}) also talks about stuff}.

The same without further text in the footnote\footnote{\cite{gauthierReligionModernityGlobalisation2019}}.

You can also reference other (sub) chapters from your document\footnote{See for example \hyperref[using-images]{Using Images}}.
This goes backwards \textbf{and} forwards, so you can reference chapters after the current one, like this\footnote{\hyperref[exporting-your-document]{Exporting your document}}. For this to work, the sections that are referenced must be labeled correctly, like this: 

\texttt{\textbackslash{section\{Using images\}}\textbackslash{label\{using-images\}}}

If you simply create the section without the label, referencing it this way within the document will not work.

The way references are formatted in the final document depends on the chosen style in the document preamble. As you can see there, \texttt{style=apa} is used in this template.

Citations and references using literature are handled by the \texttt{.bib} (bibliography) file within the \texttt{assets} folder. It contains all your literature (in BibTeX format), which can then be referenced within the text.

It is also possible to link your Overleaf account to \emph{Zotero} and organise your bibliography this way. See \href{https://www.overleaf.com/learn/how-to/How_to_link_your_Overleaf_account_to_Mendeley_and_Zotero}{here} for further details. Keep in mind that this is unfortunately a premium function and therefore requires an Overleaf subscription.

The \texttt{References} section in the document is generated automatically, when called with the \texttt{\string\printbibliography} command. By default, it only includes literature that is actually referenced within the text. This behavior can be changed by you, should you wish to do so.

For more information about citations and references using \texttt{biblatex} (as shown in this section), see \href{https://www.overleaf.com/learn/latex/Bibliography_management_with_biblatex}{this} part of the tutorial.

\section{Exporting your document}\label{exporting-your-document}

Exporting your final document with Overleaf is really easy: Simply compile it one final time to get the most up-to-date version and click the PDF download button. Just like that, your document is on your device and ready to be handed in.

For further information on exporting your document, have a look at \href{https://docs.overleaf.com/managing-projects-and-files/downloading-a-project}{this} part of the tutorial.

\newpage{}

\printbibliography[ % Prints the bibliography
heading=bibintoc,
title={References} % title of the 'references' section, change this if necessary
]

\end{document}
